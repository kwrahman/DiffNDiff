\need{Inference}

\begin{itemize}
    \item Cluster at the level of treatment to account for within-unit correlation of the error term over time (\cite{bertrand}).
    \item Do you have a small \# of clusters? Use a clustered wild bootstrap or permutation test (\cite{Cameron}, \cite{hagemann}).
\end{itemize}

%\divider\smallskip

%\itemtag{other}
%\itemtag{more}
%\itemtag{used before (would do)}
%\itemtag{twesors}


\need{Pitfalls}
\begin{itemize}
\item Do treated and untreated units appear to be on different pre-treatment trends? You have options!
\begin{itemize}
    \item Re-weight untreated units using synthetic control (\cite{Abadie}) or inverse propensity score weighting (\cite{Hirano}).
    \item Use your knowledge of the setting to select only untreated units you think will be on a similar trend (e.g. states in the same region, rather than all states).
\end{itemize} 
    \item Are units treated at different times? This can cause problems. See \cite{abraham_estimating_2018} and \cite{goodman-bacon_difference--differences_2018}.
    \item Do you have adequate power to detect ``pre-trends'' if they are present? Check with method in \cite{roth_pre-test_2019}, Section 5.2.
    \end{itemize}



\need{Rating}

\risk{Difficulty}{2}
\risk{Fun}{5}
\risk{Validity}{3}


\need{Make it Sizzle}
\begin{itemize}
    \item Can you identify a subgroup within the treated units that was not affected by treatment? This could serve as a placebo test, and may even allow you to estimate the elusive ``triple difference'' model!
\end{itemize}




